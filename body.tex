\usetheme{metropolis}
\usepackage{appendixnumberbeamer}

\usepackage{booktabs}
\usepackage[scale=2]{ccicons}

\usepackage{pgfplots}
\usepgfplotslibrary{dateplot}

\definecolor{mText}{HTML}{353535} % Black
\definecolor{mGold}{HTML}{FFC828} % Gold
\definecolor{mMaroon}{HTML}{8F0437} % Maroon
\definecolor{mTurquoise}{HTML}{4AB7C4} % Turquoise
\setbeamercolor{normal text}{fg=mText, bg=mGold!3}
\setbeamercolor{alerted text}{fg=mMaroon}
\setbeamercolor{example text}{fg=mTurquoise}
\setbeamercolor{block title}{bg=mText!15,fg=mMaroon}
\setbeamercolor{block body}{bg=mText!10,fg=mText}
\setbeamercolor{progress bar}{fg=mMaroon, bg=mMaroon!50!black!30}
\setbeamercolor{palette primary}{fg=mGold, bg=mText}

\title{PRESENTATION TEMPLATE}
\subtitle{An example of some of \LaTeX 's included presentation features}
\date{\today}
\author{Jacob Clark}
\institute{Somewhere}
\titlegraphic{\hfill\includegraphics[height=1.5cm]{example-image-a}}

\begin{document}

\maketitle

\begin{frame}{Table of contents}
  \setbeamertemplate{section in toc}[sections numbered]
  \tableofcontents[hideallsubsections]
  \note{One can implement notes in Beamer}
\end{frame}

\section{Introduction}

\begin{frame}{A Perfect Slide}
  This is an example slide. One can use \LaTeX easily here.
  \note{Talk about how \LaTeX is pretty neat. It is, so this should not be difficult.}
\end{frame}

\section{Build Up}

\begin{frame}{Relevant definitons and work}
  The \alert{magic} happens \emph{here}.
  \note{Don't forget to acknowledge history of subject}
\end{frame}


\begin{frame}{Emphasis on a list}
  \begin{itemize}[<+- | alert@+>]
    \item \alert<3>{Bow ties are\only<3>{ extremely} cool}
    \item Fezes are cool 
  \end{itemize}
  \note{Stetsons are also cool, but that would just be silly}
\end{frame}

\begin{frame}{Blocks}
  In this theme, we have the ability to call upon three pre-defined block environments.
  \begin{columns}[T,onlytextwidth]
    \column{0.475\textwidth}
      \begin{block}{Default}
        Block content.
      \end{block}

      \begin{alertblock}{Alert}
        Block content.
      \end{alertblock}

      \begin{exampleblock}{Example}
        Block content.
      \end{exampleblock}

    \column{0.475\textwidth}

      \metroset{block=fill}

      \begin{block}{Default}
        Block content.
      \end{block}

      \begin{alertblock}{Alert}
        Block content.
      \end{alertblock}

      \begin{exampleblock}{Example}
        Block content.
      \end{exampleblock}

  \end{columns}
  \note{Sometimes, one want to place theorems or examples in blocks}
\end{frame}

\begin{frame}{References}
  We can also call upon references, if need be. 
  
  For example, Automorphic forms are pretty cool. \cite{Bump96}
  \note{Indeed they are!}
\end{frame}

\section{Conclusion}

\begin{frame}{Summary}
  The main conclusion of this presentation would appear here.
  \note{Do not forget to thank department and colaborators!}
\end{frame}

\begin{frame}[standout]
  Questions?
\end{frame}

\appendix

\begin{frame}[fragile]{Backup slides}
  Additional information one might want to reference, for questions etcetera, but not numbered or noted as part of the presentation by the theme's progress bar can be placed here.
\end{frame}

\begin{frame}[allowframebreaks]{References}

  \bibliography{presentation}
  \bibliographystyle{abbrv}

\end{frame}

\begin{frame}{Source Code}

  If you would like the source code of this presentation template, and instructions for use, it is availible at 

  \begin{center}\url{github.com/jnclark/latex-presentation}\end{center}

  You can also get the source of this theme and a demo presentation from

  \begin{center}\url{github.com/matze/mtheme}\end{center}

  The theme \emph{itself} is licensed under a
  \href{http://creativecommons.org/licenses/by-sa/4.0/}{Creative Commons
  Attribution-ShareAlike 4.0 International License}.

  \begin{center}\ccbysa\end{center}
\end{frame}

\end{document}
